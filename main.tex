\documentclass{amsart}

%\usepackage{etoolbox}
%\makeatletter
%\let\ams@starttoc\@starttoc
%\makeatother
%\makeatletter
%\let\@starttoc\ams@starttoc
%\patchcmd{\@starttoc}{\makeatletter}{\makeatletter\parskip\z@}{}{}
%\makeatother

%\usepackage[parfill]{parskip}

\usepackage[colorlinks=true,linkcolor=blue,citecolor=blue,urlcolor=blue]{hyperref}
\usepackage{bookmark}
\usepackage{amsthm,thmtools,amssymb,amsmath,amscd}

\usepackage{fancyhdr}
\usepackage{esint}

\usepackage{enumerate}

\usepackage{pictexwd,dcpic}

\usepackage{graphicx}

\swapnumbers
\declaretheorem[name=Theorem,numberwithin=section]{thm}
\declaretheorem[name=Remark,style=remark,sibling=thm]{rem}
\declaretheorem[name=Lemma,sibling=thm]{lemma}
\declaretheorem[name=Proposition,sibling=thm]{prop}
\declaretheorem[name=Definition,style=definition,sibling=thm]{defn}
\declaretheorem[name=Corollary,sibling=thm]{cor}
\declaretheorem[name=Assumption,style=remark,sibling=thm]{ass}
\declaretheorem[name=Example,style=remark,sibling=thm]{example}


\numberwithin{equation}{section}

\usepackage{cleveref}
\crefname{lemma}{Lemma}{Lemmata}
\crefname{prop}{Proposition}{Propositions}
\crefname{thm}{Theorem}{Theorems}
\crefname{cor}{Corollary}{Corollaries}
\crefname{defn}{Definition}{Definitions}
\crefname{example}{Example}{Examples}
\crefname{rem}{Remark}{Remarks}
\crefname{ass}{Assumption}{Assumptions}
\crefname{not}{Notation}{Notation}

%Symbols
\renewcommand{\~}{\tilde}
\renewcommand{\-}{\bar}
\newcommand{\bs}{\backslash}
\newcommand{\cn}{\colon}
\newcommand{\sub}{\subset}

\newcommand{\N}{\mathbb{N}}
\newcommand{\R}{\mathbb{R}}
\newcommand{\Z}{\mathbb{Z}}
\renewcommand{\S}{\mathbb{S}}
\renewcommand{\H}{\mathbb{H}}
\newcommand{\C}{\mathbb{C}}
\newcommand{\K}{\mathbb{K}}
\newcommand{\Di}{\mathbb{D}}
\newcommand{\B}{\mathbb{B}}
\newcommand{\8}{\infty}

%Greek letters
\renewcommand{\a}{\alpha}
\renewcommand{\b}{\beta}
\newcommand{\g}{\gamma}
\renewcommand{\d}{\delta}
\newcommand{\e}{\epsilon}
\renewcommand{\k}{\kappa}
\renewcommand{\l}{\lambda}
\renewcommand{\o}{\omega}
\renewcommand{\t}{\theta}
\newcommand{\s}{\sigma}
\newcommand{\p}{\varphi}
\newcommand{\z}{\zeta}
\newcommand{\vt}{\vartheta}
\renewcommand{\O}{\Omega}
\newcommand{\D}{\Delta}
\newcommand{\G}{\Gamma}
\newcommand{\T}{\Theta}
\renewcommand{\L}{\Lambda}

%Mathcal Letters
\newcommand{\cL}{\mathcal{L}}
\newcommand{\cT}{\mathcal{T}}
\newcommand{\cA}{\mathcal{A}}
\newcommand{\cW}{\mathcal{W}}

%Mathematical operators
\newcommand{\INT}{\int_{\O}}
\newcommand{\DINT}{\int_{\d\O}}
\newcommand{\Int}{\int_{-\infty}^{\infty}}
\newcommand{\del}{\partial}

\newcommand{\inpr}[2]{\left\langle #1,#2 \right\rangle}
\newcommand{\fr}[2]{\frac{#1}{#2}}
\newcommand{\x}{\times}

\DeclareMathOperator{\dive}{div}
\DeclareMathOperator{\id}{id}
\DeclareMathOperator{\pr}{pr}
\DeclareMathOperator{\Diff}{Diff}
\DeclareMathOperator{\supp}{supp}
\DeclareMathOperator{\graph}{graph}
\DeclareMathOperator{\osc}{osc}
\DeclareMathOperator{\const}{const}
\DeclareMathOperator{\dist}{dist}
\DeclareMathOperator{\loc}{loc}

%Environments
\newcommand{\Theo}[3]{\begin{#1}\label{#2} #3 \end{#1}}
\newcommand{\pf}[1]{\begin{proof} #1 \end{proof}}
\newcommand{\eq}[1]{\begin{equation}\begin{alignedat}{2} #1 \end{alignedat}\end{equation}}
\newcommand{\IntEq}[4]{#1&#2#3	 &\quad &\text{in}~#4,}
\newcommand{\BEq}[4]{#1&#2#3	 &\quad &\text{on}~#4}
\newcommand{\br}[1]{\left(#1\right)}



%Logical symbols
\newcommand{\Ra}{\Rightarrow}
\newcommand{\ra}{\rightarrow}
\newcommand{\hra}{\hookrightarrow}
\newcommand{\mt}{\mapsto}

% Aleksandrov Reflection Macros
\DeclareMathOperator{\reflectionvector}{V}
\DeclareMathOperator{\reflectionangle}{\delta}
\newcommand{\reflectionplane}[1][\reflectionvector]{\ensuremath{P_{#1}}}
\newcommand{\reflectionmap}[1][\reflectionvector]{\ensuremath{R_{#1}}}
\newcommand{\reflectionset}[2][\reflectionvector]{\ensuremath{{#2}_{#1}}}
\newcommand{\reflectionhalfspace}[1][\reflectionvector]{\ensuremath{\reflectionset[{#1}]{H}}}
\DeclareMathOperator{\vertvec}{e}
\DeclareMathOperator{\origin}{O}
\DeclareMathOperator{\radialprojection}{\pi}
\DeclareMathOperator{\height}{h}
\DeclareMathOperator{\equator}{E}
\newcommand{\ip}[2]{\ensuremath{\langle{#1},{#2}\rangle}}
\DeclareMathOperator{\intersect}{\cap}
\DeclareMathOperator{\union}{\cup}
\DeclareMathOperator{\nor}{\nu}
\DeclareMathOperator{\basepoint}{p_0}
\DeclareMathOperator{\radialdistance}{r}

%Fonts
\newcommand{\mc}{\mathcal}
\renewcommand{\it}{\textit}
\newcommand{\mrm}{\mathrm}

%Spacing
\newcommand{\hp}{\hphantom}


%\parindent 0 pt

\protected\def\ignorethis#1\endignorethis{}
\let\endignorethis\relax
\def\TOCstop{\addtocontents{toc}{\ignorethis}}
\def\TOCstart{\addtocontents{toc}{\endignorethis}}

%\usepackage[left=1in,right=1in,top=1in,bottom=1in]{geometry}
\begin{document}

\title[Ancient Solutions]
 {Ancient Solutions in Riemannian Backgrounds}

\curraddr{}
\email{}
\date{\today}

\dedicatory{}
\subjclass[2010]{}
\keywords{}

\begin{abstract}
\end{abstract}

\maketitle

\section{Overview}
\label{sec:overview}

It was conjectured in \cite{IvakiBryan} that for the mean curvature flow in a positively curved, compact background, a convex, ancient solution converges backwards in time to a totally geodesic hypersurface, and there exists at most one non-trivial convex ancient solution emanating from this totally geodesic hypersurface. One line of reasoning that is likely to lead to a proof is that if an ancient solution actually has a smooth limit (this may be shown by means of a Harnack inequality; see \cite{IvakiBryan} in the case $N=\mathbb{S}^n$), then one can linearize the flow at this limit and use some version of the theory of invariant manifolds to conclude that the ancient solution lies on the unstable manifold of its limit as $t\to-\infty$ \cite[Chapters 8, 9]{lunardi2012analytic}. In fact, since the solution is assumed to be convex, its motion is monotone, and it would have to lie on the unstable manifold corresponding to the principal eigenvalue and eigenfunction of the linearization. Since the principal eigenvalue is always simple, the unstable manifold is one-dimensional, so there is only one such orbit. The unstable manifold theorem guarantees existence and uniqueness of the ancient solution emanating from any given unstable minimal hypersurface. The uniqueness of the unstable manifold also implies that the corresponding ancient solution inherits any symmetry that its limit at $t=-\infty$ may have. In the context of nonlinear heat equations, the stable and unstable manifold theory has been developed by Escher, Pr\"{u}ss, Simonett, and others (e.g., see \cite{pruess2012invariant}). It was used for Ricci flow by Gunther, Isenberg, and Knopf, e.g., \cite{guenther2002stability}. Application to mean curvature flow should be easier since the latter flow can be written as a classical quasi-linear PDE in any tubular neighborhood over the limiting hypersurface at $t=-\infty$.

\section{Linearising Graphs}
\label{sec:linearising_graphs}

The linearisation of the flow is given in \cite[Lemma 3.5]{Harltey:/2016}. Compared with the setting there, in our situation there is no weight function: \(\Xi = 0\) and so the integral terms disappear. Moreover, we do not assume that \(M_0\) is totally umbilic. Let us introduce a little notation, and rephrase the result of \cite[Lemma 3.5]{Harltey:/2016} in this notation.

The speed function, \(F: \Gamma(T^{\ast}M \otimes TM) \to C^{\infty}(M)\) may be thought of as a function \(F : \Gamma(T^{\ast} M \otimes T^{\ast}M) \times \Gamma(T^{\ast} \odot T^{\ast}M)_+ \to C^{\infty}(M)\) via
\[
F(h, g) = F(\tr_g h)
\]
where \(\Gamma(T^{\ast} \odot T^{\ast}M)_+\) denotes the fibre bundle of positive-definite, symmetric bilinear forms. Writing
\[
\dot{F}^{ij} = \partial_{h_{ij}} F = g^{ik} \partial_{h^j_k} F
\]
we obtain a positive definite (from the parabolicity condition), symmetric bi-linear form on \(T^{\ast}M\),
\[
B_{\dot{F}} (\alpha, \beta) = \alpha_i \beta_j F^{ij}
\]
where \(\alpha = \alpha_i dx^i, \beta = \beta_i dx^i\). We can use \(B_{\dot{F}}\) to define musical isomorphisms and so raise and lower indices. In particular, denote by \(\ric_{\dot{F}}\) the trace of \(\rm\) with respect to \(B_{\dot{F}}\),
\[
(\ric_{\dot{F}})_{ij} = \dot{F}^{kl} \riem_{kilj}.
\]

Let us also define the operator
\[
\Box u = \dot{F}^{ij} \nabla^2_{ij} u.
\]

Then arguing as in \cite[Lemmas 3.1, 3,2, 3.5]{Harltey:/2016} we obtain the linearisation is the Schr\"odinger operator,
\begin{equation}
\label{eq:linearisation}
L u = \Box u + \left(\overline{\ric}_{\dot{F}}(\nu_0, \nu_0) + |\mathcal{W}|^2\right) u
\end{equation}
where derivatives and norms are computed with respect to the metric \(g_0 = F_0^{\ast} \bar{g}\) induced on \(F(M_0)\) and Levi-Civita connection \(\nabla_0\). For convenience, let set
\[
V = \left(\overline{\ric}_{\dot{F}}(\nu_0, \nu_0) + |\mathcal{W}|^2\right)
\]
for the potential.


We have a basic theorem regarding the spectrum \(\sigma = \sigma(L)\) of \(L\). Let us define,
\[
\sigma_{\pm} (L) = \{\lambda \in \sigma(L) : \pm \text{Re}(\lambda) > 0\}
\]
and
\begin{align*}
\omega_+ &= \inf \{\text{Re} \lambda : \lambda \in \sigma_+\} \\
- \omega_- &= \sup \{\text{Re} \lambda : \lambda \in \sigma_-\}.
\end{align*}

\begin{thm}
Suppose \(L\) may be written in divergence form,
\[
L u = \div\left[\left(F^{ij} \nabla_j u + u X^i\right)\partial_i\right] - g(X, \nabla u) + c u.
\]
Then the eigenvalues of \(L\) are real, and the principal eigenvalue \(\lambda_1\) is simple.
\end{thm}

The proof is standard, such as in \cite[Theorems 8.37, 8.38]{GilbargTrudinger:/2001}.

\begin{rem}
We should be able to do better than above. I think probably for the divergence part, we really only need something for which Stoke's theorem may be applied. Might be worth looking into some more to see which operators satisfy this requirement. But it does seem rather restrictive - though the results to pertain to the Mean Curvature Flow at least.
\end{rem}

\begin{rem}
But more importantly, can we obtain the same results for potentials that are semi-concave (\(V(x)\) has bounded below second derivative). Is semi-concavity important, or do we just need bounded potential \(V\)? See for example \cite[Theorem 6.5.3]{Evans:/1998} which applies to operators of our form, but with \emph{negative} potential (note that Evans takes the convention of e.g. \(L = -\Delta + V\) with \(V \geq 0\) and after multiplying by \(-1\) we get \(L = \Delta - V\)). Our potentials are positive, but is there a trick that allows us to add something to \(u\) or multiply by \(u\) by something to obtain a new equation with \(V \leq 0\)? Then we may apply \cite[Theorem 6.5.3]{Evans:/1998} to discover that the principal eigenvalue is real and simple. The main point about the sign on \(V\) seems to be in showing that if \(Lu = f\) for \(f \geq 0\), then the strong maximum principle applies to show that \(u > 0\).

It appears as though the relevant result goes by the name of the Krein-Rutman theorem. See for example \cite[Appendix C, Chapter 11]{smoller:/1983}. An example is given where the theorem is applied, but again it only applies to negative potentials!
\end{rem}

If some version of the remark above is true, the resulting theorem would then become,
\begin{thm}[Provisional Theorem]
Let \(L u = \Box u + V u\) with \(V\) having bounded below second derivative (or perhaps just bounded \(V\)?). Then the principal eigenvalue \(\lambda_1\) is real and simple. That is, there exists a real eigenvalue \(\lambda_1\) for which all other eigenvalues \(\lambda \in \C\) satisfy,
\[
\text{Re} \lambda \leq \lambda_1,
\]
and the eigenspace associated to \(\lambda_1\) is one-dimensional.
\end{thm}

Next, we have some theorems regarding the linear stability of fixed points of the flow. Let \(M_{-\infty} = F_{-\infty}(M)\) denote a fixed point of the flow.

\begin{thm}
The flow admits a non-static ancient solution \(M_t\) with \(M_t \to M_{-\infty}\) (in some appropriate topology, at least something like \(h^{1+\alpha}\) - just anything better than \(C^1\) probably) if and only if
\[
\sup \{\text{Re}\lambda \in \sigma(L)\} > 0.
\]
Moreover if there is a unique eigenvalue \(\lambda\) such that \(\text{Re}(\lambda) > 0\), \(M_t\) lies locally on side of \(M_0\) (i.e. is the normal graph of a function, either strictly positive or strictly negative) and is unique up to choice of side.
\end{thm}

\begin{rem}
In the proof that follows we need to use that solutions close to \(M_{-\infty}\) are contained in the centre manifold. What conditions are required here?
\end{rem}

\begin{proof}
\cite[Theorem 9.1.3]{lunardi2012analytic} asserts that if \(\sup \{\text{Re}\lambda \in \sigma(L)\} > 0\), and \(\omega_+ > 0\) then there exists a non-static ancient solution with \(M_t \to M_{-\infty}\). For the operators we consider, the eigenvalues are discrete hence \(\omega_+ > 0\).

On the other hand, if such a solution exists, then there is \(t_0 < 0\) such that \(M_t\) lies in the unstable-centre manifold for \(t \leq t_0\) (is this \cite[Theorem 9.1.4]{lunardi2012analytic}?). But if \(\sigma_+ = \emptyset\), then \(M_{-\infty}\) is linearly stable which implies in particular that for any neighbourhood, since there is a \(t_0\) such \(M_t\) lies in this neighbourhood for \(t \leq t_0\), it remains in this neighbourhood for all \(t \geq t_0\). Thus \(M_t\) lies in every neighbourhood of \(M_{-\infty}\) and hence \(M_t = M_{-\infty}\) for all \(t\) - a contradiction (this can probably be rephrased without a contradiction argument).

Finally, suppose \(\lambda_1\) is the unique eigenvalue with positive real part. Then \cite[Theorem 9.1.3]{lunardi2012analytic} assures us that there exists at least one non-static ancient solution \(z(t)\) limiting back to \(M_0\) (i.e. \(z \to 0\) as \(t \to -\infty\). Denote by \(P : X \to V_1\) the projection of our function space \(X\) onto the eigenspace associated to \(\lamda_1\) which is one-dimensional by the theorems above. Let \(z_0^+\) be any basis for \(V_1\) with \(z_0^+ > 0\) (by the theorems above about the first eigenfunction) and \(z_0^- < 0\) also a basis. Then in fact, the proof of \cite[Theorem 9.1.3]{lunardi2012analytic} assures us of at least two ancient solutions \(z^{\pm}\) limiting back to \(0\) with \(\pm z^{\pm} > 0\).

Write \(P z^{\pm}(t) = c(t) z_0^{\pm}\) with \(c \to 0\) as \(t \to -\infty). Since \(0\) is a barrier for the flow and \(c(t) z_0^{\pm}\) is either strictly positive or strictly negative, we may assume \(c > 0\). Continuity then implies that \(c\) maps onto \((0, \epsilon)\) for some \(\epsilon > 0\).

Thus if \(v(t)\) is any other ancient solution limiting back to \(0\), then there exists \(t_0,t_1\) with \(P v(t_0) = \pm d_0 z_0 = P z(t_1)\), \(0 < d_0 < \epsilon\) for either \(z_0 = z_0^+, z = z^+\) or \(z_0 = z_0^-, z = z^-\). Thus both \(z\) and \(y\) satisfy the same equation with the same projected initial conditions (up to time shift) and hence agree by the uniqueness as in the proof of \cite[Theorem 9.1.3]{lunardi2012analytic}.
\end{proof}

\section{Linearising Immersions}
\label{sec:linearising_immersions}

The typical approach here seems to be as follows: Let \(M \subset \bar{M}\) be a hypersurface (what about higher co-dimension?) and consider graphs
\[
M_f = \{\exp_x(f(x)\nu(x) : x \in M\}
\]
where the exponential map is the ambient exponential map. Now the center manifold is a sub-manifold of \(C^{\infty}(M, \R)\) - or perhaps rather on some suitable, less-regular space such as little H\"older spaces. Little H\"older norms and so forth are defined on this space and \(M\) corresponds to the zero graph.

Suppose instead we took the parametric approach. Let
\[
F_0 : M \to \bar{M}
\]
be an immersion. Replace \(C^{\infty}(M, \R)\) with \(C^{\infty} (M, \bar{M})\). The \(C^0\) topology is induced by Haussdorf distance - we have a sort of a norm here
\[
|F(x)| = d(F_0(x), F(x)).
\]
It's not a norm because we can't add $F_1 + F_2$. Is this a major issue?

We do have norms on higher derivatives,
\[
|\nabla^k F (x)| = |\nabla^k F(x) - \nabla^k F_0(x)|
\]
Does this make sense? What exactly do I mean by \(\nabla^k\)? 

We can integrate and so forth to obtain \(L^p\) norms, including \(L^{\infty}\). Not sure about H\"older norms. 

I see now what I'm doing. This makes sense in the context of uniform spaces. This is the same as for Nash-Moser Theorem. I can't remember my history now, but I recall the idea there is similar that you're not working with a vector space but the required technical tools such as inverse function theorem carry over to this setting. See Hamilton's paper on this for details (if I recall correctly). Is this too much machinery just to avoid working with graphs? I'm always suspicious of graphs as being too restrictive.

The upshot is I would like to know if the general theory on \(C^{\infty}(M, \R)\) (or whatever appropriate weaker function space) may be carried over to maps \(F : M \to \bar{M}\) instead. this includes the graph case by
\[
F(x) = \exp_{F_0(x)} (f(x) \nu_0(x)).
\]

There are plenty of good functional properties of these uniform spaces and I think a lot of the required functional analysis carries over to this setting. Perhaps this approach is for another day though when I have the energy to work it all out. Especially since \(F\) being \(C^1\) close to \(F_0\) is any reasonable sense probably means that \(F\) can be written as a graph over \(F_0\) as above anyway!

\bibliographystyle{amsplain}
\bibliography{Bibliography.bib}
\end{document}
